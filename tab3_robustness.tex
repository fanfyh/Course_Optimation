\begin{table}[!htbp]
\footnotesize
\centering
\caption{更换被解释变量和子样本回归结果}\label{tab:robustness}
\begin{threeparttable}
\begin{tabular}{l*{5}{Y}}
\toprule
 & \multicolumn{5}{c}{\textbf{因变量}} \\
\cmidrule(lr){2-6}
变量 & 私营企业注册数量 & 每万人企业注册数量 & 企业注册数量增长率 & \makecell{删除副省级\\城市样本} & \makecell{删除扩权强县\\样本} \\
 & (1) & (2) & (3) & (4) & (5) \\
\midrule
省直管县 & 0.060\tnote{***} & 0.305\tnote{**} & 0.030\tnote{*} & 0.078\tnote{***} & 0.107\tnote{***} \\
 & (0.015) & (0.149) & (0.017) & (0.016) & (0.019) \\
\addlinespace[2pt]
时间趋势 & 控制 & 控制 & 控制 & 控制 & 控制 \\
固定效应 & 个体、时间 & 个体、时间 & 个体、时间 & 个体、时间 & 个体、时间 \\
\midrule
观测数 & 24,296 & 24,267 & 24,261 & 22,741 & 15,623 \\
R平方 & 0.910 & 0.822 & 0.121 & 0.898 & 0.888 \\
\bottomrule
\end{tabular}

\begin{tablenotes}
\scriptsize
\item 注:括号内为县级聚类稳健标准误。***、**、*分别表示1%、5%、10%显著性水平。
\item 控制变量包含:人均GDP对数、人口规模对数、财政自给度、产业结构等事前特征时间趋势项。
\item 列(4)排除副省级城市样本,列(5)排除实施扩权强县改革的样本。
\end{tablenotes}
\end{threeparttable}
\end{table}

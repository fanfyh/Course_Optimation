\begin{table}[!htbp]
\footnotesize
\centering
\caption{倾向得分匹配双重差分方法结果}\label{tab:psmdid}
\begin{threeparttable}
\begin{tabular}{lccccc}
\toprule
 & \multicolumn{5}{c}{\textbf{分类变量}} \\
\cmidrule(lr){2-6}
变量 & 全样本 & 县级市 & 粮食主产县 & 国家贫困县 & 省边界县 \\
 & (1) & (2) & (3) & (4) & (5) \\
\midrule
省直管县 & 0.067\tnote{***} & 0.054\tnote{**} & 0.056\tnote{**} & 0.058\tnote{***} & 0.066\tnote{***} \\
 & (0.016) & (0.018) & (0.017) & (0.017) & (0.018) \\
\midrule
控制变量 & 基础模型 & 分类匹配 & 分类匹配 & 分类匹配 & 分类匹配 \\
时间趋势 & \multicolumn{5}{c}{控制} \\
固定效应 & \multicolumn{5}{c}{个体、时间} \\
\cmidrule(lr){1-1}
观测数 & 17,300 & 16,752 & 16,426 & 16,551 & 16,479 \\
R平方 & 0.894 & 0.895 & 0.889 & 0.887 & 0.887 \\
\bottomrule
\end{tabular}

\begin{tablenotes}
\scriptsize
\item[注]:***、**、*分别表示在1%、5%、10%水平上显著,括号内为县级聚类稳健标准误。
\item 控制变量包括:人均GDP对数、人口规模对数、财政自给度等时间趋势项。列(2)-(5)分别实施基于县级市、粮食主产县、国家贫困县和省边界县的倾向得分匹配。
\end{tablenotes}
\end{threeparttable}
\end{table}

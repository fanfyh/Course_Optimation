\newcolumntype{R}{>{\raggedleft\arraybackslash}X} % 右对齐列

\begin{table}[htbp]
\footnotesize
\centering
\caption{Goodman-Bacon分解结果}\label{tab:goodmanbacon}
\begin{threeparttable}
\begin{tabular}{l l l R R}
\toprule
组别 & 处理组 & 对照组 & 权重 & 平均处理效应 \\
& & & (\%) & (1) \\ 
\midrule
组1 & 处理组 & 从未接受处理组 & 75.64 & 0.0968\tnote{a} \\
& & & (0.75) & (0.010) \\
\addlinespace[0.5ex]
组2 & 早处理组 & 晚处理组受处理前 & 13.94 & 0.0189 \\
& & & (0.31) & (0.007) \\
\addlinespace[0.5ex]
组3 & 晚处理组 & 早处理组受处理后 & 10.42 & 0.0124 \\
& & & (0.24) & (0.005) \\
\bottomrule
\end{tabular}

\begin{tablenotes}
\scriptsize
\item[a] 标准差经县级聚类调整。数据来源:《中国县域统计年鉴》2000-2013。分解方法参考Goodman-Bacon(2021)的时变处理效应模型。
\item 括号内为自助法标准差(1000次重复抽样),加权平均总效应0.063。
\end{tablenotes}
\end{threeparttable}
\end{table}






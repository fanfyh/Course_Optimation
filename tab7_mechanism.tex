\begin{table}[!htbp]
\footnotesize
\centering
\caption{省直管县的财政治理方式影响}\label{tab:mechanism}
\begin{threeparttable}
\begin{tabular}{lccccc}
\toprule
变量 & (1) & (2) & (3) & (4) & (5) \\
 & 税收分成比例 & 人均转移支付 & 税收征管力度 & 生产性支出 & 民生性支出 \\
\midrule
省直管县 & 0.014\tnote{**} & 0.042\tnote{***} & 0.027 & 0.126 & 0.015 \\
 & (0.006) & (0.015) & (0.022) & (0.079) & (0.014) \\
\addlinespace[0.5ex]
税收分成比例 & & & 0.476\tnote{***} & -0.183 & 0.189\tnote{***} \\
 & & & (0.048) & (0.179) & (0.033) \\
\addlinespace[0.5ex]
人均转移支付 & & & -0.056\tnote{***} & 0.884\tnote{***} & 0.248\tnote{***} \\
 & & & (0.014) & (0.064) & (0.016) \\
\midrule
观测数 & 12,203 & 12,187 & 12,186 & 12,187 & 12,187 \\
R平方 & 0.854 & 0.966 & 0.744 & 0.809 & 0.945 \\
控制变量 & \multicolumn{5}{c}{时间趋势项、个体固定效应} \\
固定效应 & \multicolumn{5}{c}{个体、时间双重固定效应} \\
\bottomrule
\end{tabular}

\begin{tablenotes}
\scriptsize
\item 注:括号内为县级聚类标准误,***、**、*分别表示1%、5%、10%显著性水平。
\item 样本区间为2000-2006年,覆盖全国县级财政单位。
\item 第(1)(2)列为第一阶段回归,展示改革对税收分成比例和转移支付的直接影响。
\item 第(3)-(5)列第二至第四阶段回归,检验税收分成和转移支付的中介效应。
\end{tablenotes}
\end{threeparttable}
\end{table}

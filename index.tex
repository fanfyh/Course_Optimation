% templates/cufe-template.latex
\documentclass{beamer}
\usepackage{ctex}
\usepackage{cufe}
\usepackage[T1]{fontenc}
\usepackage{xcolor}
\usepackage{tcolorbox}
\usepackage{bookmark}
\tcbuselibrary{skins, breakable}  % 添加必要库
% 添加至模板开头
\makeatletter
\newcommand{\tightlist}{%
  \setlength{\itemsep}{0pt}\setlength{\parskip}{0pt}}
\makeatother

% 修复enhanced jigsaw为enhancedjigsaw
\tcbset{
    mycallout/.style={
        enhancedjigsaw,  % 正确键名(无空格)
        breakable,
        colback=blue!5,
        colframe=blue!50!black
    }
}

% other packages
\usepackage{latexsym,amsmath,xcolor,multicol,booktabs,calligra}
\usepackage{graphicx,pstricks,listings,stackengine}
\title{经济学中的优化方法}
\subtitle{}
\author{范翻}
\institute{中央财经大学(CCFD)}
\date{}
% ==== 加载自定义设置 ====
% \input{templates/cufe-settings.sty}  % 可选,用于集中存放样式
% defs

\definecolor{deepblue}{rgb}{0,0,0.5}
\definecolor{deepred}{rgb}{0.6,0,0}
\definecolor{deepgreen}{rgb}{0,0.5,0}
\definecolor{halfgray}{gray}{0.55}

\lstset{
    basicstyle=\ttfamily\small,
    keywordstyle=\bfseries\color{deepblue},
    emphstyle=\ttfamily\color{deepred},    % Custom highlighting style
    stringstyle=\color{deepgreen},
    numbers=left,
    numberstyle=\small\color{halfgray},
    rulesepcolor=\color{red!20!green!20!blue!20},
    frame=shadowbox,
}
\begin{document}

% ==== 自动插入标题页 ==== 
\kaishu  % 楷书样式
\begin{frame}
  \titlepage
  \begin{figure}[htpb]
    \centering
    \includegraphics[width=0.2\linewidth]{pic/cufe_logo_blue.eps}
  \end{figure}
\end{frame}



% ==== Quarto生成的内容 ====
\begin{frame}
Date: 周二 14:00-16:40

Room: \textbf{沙河主教 401}

Instructors: 范翻
\end{frame}

\begin{frame}[fragile]
\begin{tcolorbox}[enhanced jigsaw, colback=white, left=2mm, title=\textcolor{quarto-callout-note-color}{\faInfo}\hspace{0.5em}{This workshop template}, toprule=.15mm, toptitle=1mm, breakable, bottomtitle=1mm, opacitybacktitle=0.6, leftrule=.75mm, opacityback=0, coltitle=black, colframe=quarto-callout-note-color-frame, titlerule=0mm, colbacktitle=quarto-callout-note-color!10!white, rightrule=.15mm, bottomrule=.15mm, arc=.35mm]

This workshop template contains 4 pages:

\begin{itemize}
\item
  Home: \texttt{index.qmd} (this page)
\item
  About: \texttt{about.qmd}
\end{itemize}

Two content pages

\begin{itemize}
\item
  Page without code: \texttt{part\_1\_prep.qmd}
\item
  Page with R code: \texttt{part\_2\_eda.qmd}
\end{itemize}

It is straightforward to add more content pages: you need to create a
new \texttt{.qmd} file (or copy/paste the existing ones), then link the
new page inside \texttt{\_quarto.yml}.

\end{tcolorbox}

\begin{tcolorbox}[enhanced jigsaw, colback=white, left=2mm, title=\textcolor{quarto-callout-tip-color}{\faLightbulb}\hspace{0.5em}{Homepage of your workshop}, toprule=.15mm, toptitle=1mm, breakable, bottomtitle=1mm, opacitybacktitle=0.6, leftrule=.75mm, opacityback=0, coltitle=black, colframe=quarto-callout-tip-color-frame, titlerule=0mm, colbacktitle=quarto-callout-tip-color!10!white, rightrule=.15mm, bottomrule=.15mm, arc=.35mm]

This is the \textbf{homepage} \texttt{index.qmd} for your workshop, so
ideally it should contain some key information such as \textbf{time and
place, instructors} and information of the course/workshop.

It should be easy to navigate.

\end{tcolorbox}
\end{frame}

\section{Welcome!}\label{welcome}

\begin{frame}{Welcome!}
\begin{itemize}
\tightlist
\item
  The goal of the workshop is to \ldots{} (\emph{insert your message})
\item
  For example, introduce kep concepts in machine learning, such as
  regularisation.
\item
  Workshop material can be found in the workshop
  \href{https://github.com}{github repository}.
\end{itemize}

\begin{block}{Learning Objectives}
\phantomsection\label{learning-objectives}
At the end of the tutorial, participants will be able to

\begin{itemize}
\tightlist
\item
  understand key concepts for \ldots{} (\emph{insert your message})
\item
  For example, training machine learning models such as regularisation.
\item
  another objective
\end{itemize}
\end{block}

\begin{block}{Pre-requisites}
\phantomsection\label{pre-requisites}
\begin{itemize}
\tightlist
\item
  Basic familiarity with R
\item
  Some other knowledge
\end{itemize}
\end{block}
\end{frame}

\section{Schedule}\label{schedule}

\begin{frame}{Schedule}
\begin{tcolorbox}[enhanced jigsaw, colback=white, left=2mm, title=\textcolor{quarto-callout-tip-color}{\faLightbulb}\hspace{0.5em}{Tabular schedule}, toprule=.15mm, toptitle=1mm, breakable, bottomtitle=1mm, opacitybacktitle=0.6, leftrule=.75mm, opacityback=0, coltitle=black, colframe=quarto-callout-tip-color-frame, titlerule=0mm, colbacktitle=quarto-callout-tip-color!10!white, rightrule=.15mm, bottomrule=.15mm, arc=.35mm]

It can be useful to include a tabular schedule with links.

\end{tcolorbox}

\begin{longtable}[]{@{}
  >{\centering\arraybackslash}p{(\columnwidth - 4\tabcolsep) * \real{0.3333}}
  >{\centering\arraybackslash}p{(\columnwidth - 4\tabcolsep) * \real{0.3333}}
  >{\centering\arraybackslash}p{(\columnwidth - 4\tabcolsep) * \real{0.3333}}@{}}
\toprule\noalign{}
\begin{minipage}[b]{\linewidth}\centering
Time
\end{minipage} & \begin{minipage}[b]{\linewidth}\centering
Topic
\end{minipage} & \begin{minipage}[b]{\linewidth}\centering
Presenter
\end{minipage} \\
\midrule\noalign{}
\endhead
9:00 - 10:30 & \href{part_1_prep.qmd}{Session 1: Preparation} &
Instructor A \\
10:45 - 12:00 & \href{part_2_eda.qmd}{Session 2: Exploratory Analysis} &
Instructor B, C \\
\bottomrule\noalign{}
\end{longtable}
\end{frame}

\end{document}

\begin{table}[htbp]
\footnotesize
\centering
\caption{市县经济实力异质性分析结果}\label{tab:hetero_eco}
\begin{threeparttable}
\begin{tabular}{lcccc}
\toprule
 & \multicolumn{4}{c}{\textbf{县域创业活力}} \\
\cmidrule(lr){2-5}
变量 & (1) & (2) & (3) & (4) \\
 & 市GDP & 市排名 & 县GDP & 县排名 \\
\midrule
省直管县 & 0.526\tnote{**} & 0.052 & 0.856\tnote{***} & 0.007 \\
 & (0.220) & (0.029) & (0.186) & (0.027) \\
省直管县$\times$经济指标 & -0.051\tnote{**} & 0.059 & -0.093\tnote{***} & 0.153\tnote{***} \\
 & (0.025) & (0.044) & (0.022) & (0.045) \\
\midrule
观测数 & 24,297 & 24,297 & 24,297 & 24,297 \\
R平方 & 0.896 & 0.896 & 0.896 & 0.896 \\
\bottomrule
\end{tabular}

\begin{tablenotes}
\scriptsize
\item[注]:括号内为县级聚类标准误,***、**、*分别表示1%、5%、10%显著性水平。
\item (1)(2)列考察市级经济实力调节效应,(3)(4)列考察县级经济实力调节效应。
\item 经济指标归一化处理: 市/县GDP取对数,排名为省级百分位数。
\end{tablenotes}
\end{threeparttable}
\end{table}

\begin{table}[htbp]
\footnotesize
\centering
\caption{税收征管力度与财政支出的创业驱动效应}\label{tab:mechanism}
\begin{threeparttable}
\begin{tabular}{l*{4}{c}}
\toprule
 & \multicolumn{4}{c}{\textbf{因变量}} \\
\cmidrule(lr){2-5}
变量 & (1) 税收征管力度 & (2) 生产性支出 & (3) 民生性支出 & (4) 县域创业活力 \\
\midrule
省直管县 & 0.031 & 0.161\tnote{**} & 0.028\tnote{*} & 0.057\tnote{**} \\
 & (0.022) & (0.084) & (0.015) & (0.037) \\
\addlinespace[2pt]
税收征管力度 & — & — & — & 0.008 \\
 &  &  &  & (0.025) \\
生产性支出 & — & — & — & 0.015\tnote{***} \\
 &  &  &  & (0.005) \\
民生性支出 & — & — & — & 0.050\tnote{*} \\
 &  &  &  & (0.029) \\
\midrule
时间趋势 & 控制 & 控制 & 控制 & 控制 \\
固定效应 & 个体、时间 & 个体、时间 & 个体、时间 & 个体、时间 \\
观测数 & 12,202 & 12,202 & 12,202 & 12,202 \\
R平方 & 0.736 & 0.802 & 0.941 & 0.903 \\
\bottomrule
\end{tabular}

\begin{tablenotes}
\scriptsize
\item 注:括号内为县级聚类稳健标准误。***、**、*分别表示1%、5%、10%显著性水平。
\item 时间趋势包含:人均GDP、人口规模等社会经济变量的二次项趋势。
\item 财政支出口径调整导致样本期间为2000-2006年,数据来源为《全国地市县财政统计资料汇编》。
\end{tablenotes}
\end{threeparttable}
\end{table}

% templates/cufe-template.latex
\documentclass{beamer}
\usepackage{ctex}
\usepackage{cufe}
\usepackage[T1]{fontenc}
\usepackage{xcolor}
\usepackage{tcolorbox}
\usepackage{bookmark}
\usepackage{booktabs}
\usepackage{threeparttable}
\usepackage{makecell}
\usepackage{adjustbox}
% 在导言区添加
\usepackage{array}
\newcolumntype{R}{>{\raggedleft\arraybackslash}X}

\tcbuselibrary{skins, breakable}  % 添加必要库
% 添加至模板开头
\makeatletter
\newcommand{\tightlist}{%
  \setlength{\itemsep}{0pt}\setlength{\parskip}{0pt}}
\makeatother

% 修复enhanced jigsaw为enhancedjigsaw
\tcbset{
    mycallout/.style={
        enhancedjigsaw,  % 正确键名(无空格)
        breakable,
        colback=blue!5,
        colframe=blue!50!black
    }
}

% other packages
\usepackage{latexsym,amsmath,xcolor,multicol,booktabs,calligra}
\usepackage{graphicx,pstricks,listings,stackengine}
\title{省以下财政体制改革如何激发县域创业活力}
\subtitle{基于财政治理方式的机制分析}
\author{范翻}
\institute{中国财政发展协同创新中心}
\date{2025-04-15}
% ==== 加载自定义设置 ====
% \input{templates/cufe-settings.sty}  % 可选,用于集中存放样式
% defs

\definecolor{deepblue}{rgb}{0,0,0.5}
\definecolor{deepred}{rgb}{0.6,0,0}
\definecolor{deepgreen}{rgb}{0,0.5,0}
\definecolor{halfgray}{gray}{0.55}

\lstset{
    basicstyle=\ttfamily\small,
    keywordstyle=\bfseries\color{deepblue},
    emphstyle=\ttfamily\color{deepred},    % Custom highlighting style
    stringstyle=\color{deepgreen},
    numbers=left,
    numberstyle=\small\color{halfgray},
    rulesepcolor=\color{red!20!green!20!blue!20},
    frame=shadowbox,
}
\begin{document}

% ==== 自动插入标题页 ==== 
\kaishu  % 楷书样式
\begin{frame}
  \titlepage
  \begin{figure}[htpb]
    \centering
    \includegraphics[width=0.2\linewidth]{pic/cufe_logo_blue.eps}
  \end{figure}
\end{frame}



% ==== Quarto生成的内容 ====
\section{研究背景}\label{ux7814ux7a76ux80ccux666f}

\subsection{问题缘起}\label{ux95eeux9898ux7f18ux8d77}

\begin{frame}{问题缘起}
\begin{itemize}
\tightlist
\item
  随着经济高质量发展的推动,创业活动成为推动就业和创新的核心力量。
\item
  近年来,我国出台了一系列支持创业的政策,其中商事制度改革和``双创''政策尤为突出。
\item
  然而,现有研究主要集中在政策工具上,较少关注基础性的财政体制对创业的影响。
\end{itemize}
\end{frame}

\subsection{研究价值}\label{ux7814ux7a76ux4ef7ux503c}

\begin{frame}{研究价值}
\begin{itemize}
\tightlist
\item
  财税制度改革,尤其是省直管县改革,作为国家治理的重要组成部分,如何影响地方政府的行为,从而激活县域的创业活力,是本研究的核心问题。
\item
  本文探讨了如何通过财政体制改革,特别是通过优化资源配置和提升地方政府财政自主性,来激发地方创业活力。
\end{itemize}
\end{frame}

\subsection{1994年分税制改革}\label{ux5e74ux5206ux7a0eux5236ux6539ux9769}

\begin{frame}{1994年分税制改革}
\begin{itemize}
\tightlist
\item
  \textbf{目标}:通过中央财权集中推动区域协调发展,优化政府间财政关系。
\item
  \textbf{效果}:

  \begin{itemize}
  \tightlist
  \item
    地方财政自给率持续下降,地方财政承压。
  \item
    省以下财政体制逐级模仿中央-地方分成模式,形成``市卡县''\,``市压县''的层级挤压效应。
  \item
    基层政府面临``支出责任与财力错配''的问题,公共服务供给不足,县域经济活力受到严重制约。
  \end{itemize}
\end{itemize}
\end{frame}

\subsection{省直管县改革}\label{ux7701ux76f4ux7ba1ux53bfux6539ux9769}

\begin{frame}{省直管县改革}
\begin{itemize}
\tightlist
\item
  \textbf{背景}:为破解基层财政困境,2004年起实施省直管县改革,增强县级政府财政自主权。
\item
  \textbf{改革内容}:

  \begin{itemize}
  \tightlist
  \item
    通过重构税收分成机制,提升县级政府财政自主权。
  \item
    改革直接联系省与县的财政,优化资源配置,提高基层政府的财政自主性。
  \end{itemize}
\item
  \textbf{目标}:提升基层财政自主权,改善县级政府的税收征管能力,优化公共服务供给,提升县域创业活力。
\end{itemize}
\end{frame}

\section{文献综述与研究假说}\label{ux6587ux732eux7efcux8ff0ux4e0eux7814ux7a76ux5047ux8bf4}

\subsection{1. 创业驱动因素}\label{ux521bux4e1aux9a71ux52a8ux56e0ux7d20}

\begin{frame}{1. 创业驱动因素}
\begin{itemize}
\tightlist
\item
  \textbf{个人特征}:早期研究聚焦于企业家个人特质,如风险偏好、人力资本等。
\item
  \textbf{宏观经济环境}:商业周期波动对资源分配和创业风险的影响。
\item
  \textbf{制度环境}:制度环境的稳定性和透明度对创业活力的支撑作用。

  \begin{itemize}
  \tightlist
  \item
    \textbf{相关文献}:Dai 和 Si (2018),Clark 和 Ramachandran
    (2019),Sahasranamam 和 Nandakumar (2020)。
  \end{itemize}
\end{itemize}
\end{frame}

\subsection{2.
省直管县的经济效应}\label{ux7701ux76f4ux7ba1ux53bfux7684ux7ecfux6d4eux6548ux5e94}

\begin{frame}{2. 省直管县的经济效应}
\begin{itemize}
\tightlist
\item
  \textbf{财政体制改革}:通过改变税收征管和公共支出结构,省直管县改革可能促进地方创业。
\item
  \textbf{地方财政自主权}:增强地方财政自给能力,改善税收政策和公共服务供给,提升创业环境。
\item
  \textbf{已有研究}:Li等(2016),宋恒等(2024)讨论了改革对资源配置和经济增长的影响。
\end{itemize}
\end{frame}

\subsection{3.
省直管县的财政治理方式}\label{ux7701ux76f4ux7ba1ux53bfux7684ux8d22ux653fux6cbbux7406ux65b9ux5f0f}

\begin{frame}{3. 省直管县的财政治理方式}
\begin{itemize}
\tightlist
\item
  \textbf{税收分成和转移支付}:

  \begin{itemize}
  \tightlist
  \item
    \textbf{税收分成}:通过提高税收分成比例,增强地方政府税收自主权,推动税收征管力度。
  \item
    \textbf{转移支付}:通过扩大转移支付,提升地方政府财政支出能力,影响公共服务供给。
  \end{itemize}
\item
  \textbf{两种财政治理方式的不同作用}:

  \begin{itemize}
  \tightlist
  \item
    税收分成提升:增强地方税收征管力度。
  \item
    转移支付扩张:可能削弱税收征管力度,但加强公共服务供给。
  \end{itemize}
\end{itemize}
\end{frame}

\subsection{研究假说}\label{ux7814ux7a76ux5047ux8bf4}

\begin{frame}{研究假说}
\begin{itemize}
\tightlist
\item
  \textbf{假说1}:省直管县改革会提升地区创业活力。
\item
  \textbf{假说2}:省直管县对税收征管力度的影响方向不明确。
\item
  \textbf{假说3a}:省直管县会通过增加生产性支出促进创业活力。
\item
  \textbf{假说3b}:省直管县会通过增加民生性支出促进创业活力。
\end{itemize}
\end{frame}

\section{研究设计}\label{ux7814ux7a76ux8bbeux8ba1}

\subsection{模型设计与变量说明}\label{ux6a21ux578bux8bbeux8ba1ux4e0eux53d8ux91cfux8bf4ux660e}

\begin{frame}{模型设计与变量说明}
\begin{itemize}
\tightlist
\item
  本研究采用多期双重差分(DID)模型来检验省直管县改革对县域创业活力的影响。
\end{itemize}

\[
Y_{it} = \alpha + \beta \cdot \text{Post}_{it} + \gamma X_{it} + \mu_i + \lambda_t + \varepsilon_{it}
\]

\begin{itemize}
\tightlist
\item
  \(Y_{it}\):被解释变量,表示县域创业活力的度量。
\item
  \(\text{Post}_{it}\):处理变量,表示县在年份\(t\)是否实施了省直管县改革。
\item
  \(X_{it}\):控制变量,包括人均GDP、人口规模、财政自主度等。
\item
  \(\mu_i\),\(\lambda_t\):县级和年份固定效应。
\item
  \(\varepsilon_{it}\):随机误差项。
\end{itemize}
\end{frame}

\subsection{主要变量及描述性统计}\label{ux4e3bux8981ux53d8ux91cfux53caux63cfux8ff0ux6027ux7edfux8ba1}

\begin{frame}{主要变量及描述性统计}
\[
\begin{array}{|c|c|c|c|c|}
\hline
\text{变量名} & \text{观测数} & \text{均值} & \text{最小值} & \text{最大值} \\
\hline
\text{县域创业活力} & 24,297 & 4.95 & 0.00 & 9.12 \\
\text{私营企业注册数量对数} & 24,297 & 4.62 & 0.00 & 9.11 \\
\text{每万人新注册企业数量} & 24,297 & 5.78 & 0.00 & 217.00 \\
\text{企业注册数量增长率} & 24,261 & 0.21 & -0.69 & 2.40 \\
\text{“省直管县”改革} & 24,297 & 0.22 & 0.00 & 1.00 \\
\text{“扩权强县”改革} & 24,297 & 0.21 & 0.00 & 1.00 \\
\text{人均地区生产总值对数} & 24,297 & 8.94 & 6.51 & 12.57 \\
\text{人口规模对数} & 24,297 & 12.77 & 8.99 & 14.70 \\
\text{财政自主度} & 24,296 & 0.33 & 0.03 & 1.09 \\
\text{产业结构} & 24,296 & 0.72 & 0.37 & 1.00 \\
\text{人口密度} & 24,297 & 5.05 & 0.49 & 8.04 \\
\hline
\end{array}
\]
\end{frame}

\subsection{样本选择与数据来源}\label{ux6837ux672cux9009ux62e9ux4e0eux6570ux636eux6765ux6e90}

\begin{frame}{样本选择与数据来源}
\begin{itemize}
\tightlist
\item
  \textbf{数据来源}:

  \begin{itemize}
  \tightlist
  \item
    \textbf{工商企业注册数据}:2000-2013年中国工商企业注册数据库,包含企业名称、注册地址、成立时间、企业类型等。
  \item
    \textbf{社会经济数据}:来自《中国县域统计年鉴》(2000-2013)、《中国城市统计年鉴》(2000-2013)等。
  \end{itemize}
\item
  \textbf{样本选择}:选取1745个县(市)作为研究对象,样本总观测值为24,297,其中887个县(市)实施了省直管县改革,858个县(市)未实施改革。
\end{itemize}
\end{frame}

\section{实证结果}\label{ux5b9eux8bc1ux7ed3ux679c}

\subsection{基准回归结果}\label{ux57faux51c6ux56deux5f52ux7ed3ux679c}

\begin{frame}{基准回归结果}
\begin{adjustbox}{max width=0.95\textwidth}
\begin{table}[!htbp]
  \footnotesize
  \centering
  \caption{基准回归结果}\label{tab:baseline}
  \begin{threeparttable}
  \begin{tabular}{l*{4}{c}}
  \toprule
   & \multicolumn{4}{c}{\textbf{县域创业活力指数}} \\
  \cmidrule(lr){2-5}
  变量 & (1) & (2) & (3) & (4) \\
   & 总效应 & 制造业 & 服务业 & 个体经营 \\
  \midrule
  省直管县 & 0.163\tnote{***} & 0.088\tnote{**} & 0.214\tnote{***} & 0.032 \\
   & (0.045) & (0.038) & (0.054) & (0.021) \\
  地区GDP增长率 & 0.012\tnote{*} & 0.007 & 0.017\tnote{**} & 0.003 \\
   & (0.006) & (0.005) & (0.008) & (0.003) \\
  人口密度 & 0.084\tnote{**} & 0.051 & 0.103\tnote{**} & 0.019 \\
   & (0.034) & (0.032) & (0.041) & (0.016) \\
  \addlinespace[0.5ex]
  控制变量 & 是 & 是 & 是 & 是 \\
  时间固定效应 & 是 & 是 & 是 & 是 \\
  个体固定效应 & 是 & 是 & 是 & 是 \\
  \midrule
  观测数 & 12,201 & 12,201 & 12,201 & 12,201 \\
  R平方 & 0.836 & 0.768 & 0.804 & 0.612 \\
  \bottomrule
  \end{tabular}
  
  \begin{tablenotes}
  \scriptsize
  \item 注:***、**、*分别表示1%、5%、10%显著性水平;括号内为县级聚类标准误
  \item 控制变量包含城乡收入比、工业化率等7项指标(受限版面未展示)
  \end{tablenotes}
  \end{threeparttable}
  \end{table}
  
\end{adjustbox}

\begin{itemize}
\tightlist
\item
  改革平均提升创业活力6.3\%(1\%显著),通过逐步加入控制变量验证稳健性
\end{itemize}
\end{frame}

\subsection{事件研究法}\label{ux4e8bux4ef6ux7814ux7a76ux6cd5}

\begin{frame}{事件研究法}
\begin{figure}
    \centering
    \includegraphics[width=\textwidth]{pic/图2-plot_event_study.pdf} 
    \caption{\text{事件研究法结果图}}
\end{figure}
\end{frame}

\subsection{安慰剂检验}\label{ux5b89ux6170ux5242ux68c0ux9a8c}

\begin{frame}{安慰剂检验}
\begin{figure}
    \centering
    \includegraphics[width=\textwidth]{pic/图2-plot_event_study.pdf} 
    \caption{\text{安慰剂检验结果图}}
\end{figure}
\end{frame}

\subsection{稳健性检验:替代变量与子样本}\label{ux7a33ux5065ux6027ux68c0ux9a8cux66ffux4ee3ux53d8ux91cfux4e0eux5b50ux6837ux672c}

\begin{frame}{稳健性检验:替代变量与子样本}
\vspace{-2mm}
\begin{adjustbox}{width=0.95\textwidth,center} 
\begin{table}[!htbp]
\footnotesize
\centering
\caption{更换被解释变量和子样本回归结果}\label{tab:robustness}
\begin{threeparttable}
\begin{tabular}{l*{5}{Y}}
\toprule
 & \multicolumn{5}{c}{\textbf{因变量}} \\
\cmidrule(lr){2-6}
变量 & 私营企业注册数量 & 每万人企业注册数量 & 企业注册数量增长率 & \makecell{删除副省级\\城市样本} & \makecell{删除扩权强县\\样本} \\
 & (1) & (2) & (3) & (4) & (5) \\
\midrule
省直管县 & 0.060\tnote{***} & 0.305\tnote{**} & 0.030\tnote{*} & 0.078\tnote{***} & 0.107\tnote{***} \\
 & (0.015) & (0.149) & (0.017) & (0.016) & (0.019) \\
\addlinespace[2pt]
时间趋势 & 控制 & 控制 & 控制 & 控制 & 控制 \\
固定效应 & 个体、时间 & 个体、时间 & 个体、时间 & 个体、时间 & 个体、时间 \\
\midrule
观测数 & 24,296 & 24,267 & 24,261 & 22,741 & 15,623 \\
R平方 & 0.910 & 0.822 & 0.121 & 0.898 & 0.888 \\
\bottomrule
\end{tabular}

\begin{tablenotes}
\scriptsize
\item 注:括号内为县级聚类稳健标准误。***、**、*分别表示1%、5%、10%显著性水平。
\item 控制变量包含:人均GDP对数、人口规模对数、财政自给度、产业结构等事前特征时间趋势项。
\item 列(4)排除副省级城市样本,列(5)排除实施扩权强县改革的样本。
\end{tablenotes}
\end{threeparttable}
\end{table}

\end{adjustbox}

\begin{itemize}
\tightlist
\item
  结果稳定性检验显示:删除干扰样本后改革效应更显著
\end{itemize}
\end{frame}

\subsection{稳健性检验:Goodman-Bacon分解结果}\label{ux7a33ux5065ux6027ux68c0ux9a8cgoodman-baconux5206ux89e3ux7ed3ux679c}

\begin{frame}{稳健性检验:Goodman-Bacon分解结果}
\begin{adjustbox}{max totalheight=0.7\textheight} 
\newcolumntype{R}{>{\raggedleft\arraybackslash}X} % 右对齐列

\begin{table}[htbp]
\footnotesize
\centering
\caption{Goodman-Bacon分解结果}\label{tab:goodmanbacon}
\begin{threeparttable}
\begin{tabular}{l l l R R}
\toprule
组别 & 处理组 & 对照组 & 权重 & 平均处理效应 \\
& & & (\%) & (1) \\ 
\midrule
组1 & 处理组 & 从未接受处理组 & 75.64 & 0.0968\tnote{a} \\
& & & (0.75) & (0.010) \\
\addlinespace[0.5ex]
组2 & 早处理组 & 晚处理组受处理前 & 13.94 & 0.0189 \\
& & & (0.31) & (0.007) \\
\addlinespace[0.5ex]
组3 & 晚处理组 & 早处理组受处理后 & 10.42 & 0.0124 \\
& & & (0.24) & (0.005) \\
\bottomrule
\end{tabular}

\begin{tablenotes}
\scriptsize
\item[a] 标准差经县级聚类调整。数据来源:《中国县域统计年鉴》2000-2013。分解方法参考Goodman-Bacon(2021)的时变处理效应模型。
\item 括号内为自助法标准差(1000次重复抽样),加权平均总效应0.063。
\end{tablenotes}
\end{threeparttable}
\end{table}






\end{adjustbox}

\vspace{-2mm}

\begin{itemize}
\tightlist
\item
  分解显示75.6\%的效应源自处理组vs.未处理组对比,验证DID结果可靠性
\end{itemize}
\end{frame}

\subsection{稳健性检验:PSM-DID}\label{ux7a33ux5065ux6027ux68c0ux9a8cpsm-did}

\begin{frame}{稳健性检验:PSM-DID}
\begin{adjustbox}{max totalheight=0.7\textheight} 
\begin{table}[!htbp]
\footnotesize
\centering
\caption{倾向得分匹配双重差分方法结果}\label{tab:psmdid}
\begin{threeparttable}
\begin{tabular}{lccccc}
\toprule
 & \multicolumn{5}{c}{\textbf{分类变量}} \\
\cmidrule(lr){2-6}
变量 & 全样本 & 县级市 & 粮食主产县 & 国家贫困县 & 省边界县 \\
 & (1) & (2) & (3) & (4) & (5) \\
\midrule
省直管县 & 0.067\tnote{***} & 0.054\tnote{**} & 0.056\tnote{**} & 0.058\tnote{***} & 0.066\tnote{***} \\
 & (0.016) & (0.018) & (0.017) & (0.017) & (0.018) \\
\midrule
控制变量 & 基础模型 & 分类匹配 & 分类匹配 & 分类匹配 & 分类匹配 \\
时间趋势 & \multicolumn{5}{c}{控制} \\
固定效应 & \multicolumn{5}{c}{个体、时间} \\
\cmidrule(lr){1-1}
观测数 & 17,300 & 16,752 & 16,426 & 16,551 & 16,479 \\
R平方 & 0.894 & 0.895 & 0.889 & 0.887 & 0.887 \\
\bottomrule
\end{tabular}

\begin{tablenotes}
\scriptsize
\item[注]:***、**、*分别表示在1%、5%、10%水平上显著,括号内为县级聚类稳健标准误。
\item 控制变量包括:人均GDP对数、人口规模对数、财政自给度等时间趋势项。列(2)-(5)分别实施基于县级市、粮食主产县、国家贫困县和省边界县的倾向得分匹配。
\end{tablenotes}
\end{threeparttable}
\end{table}

\end{adjustbox}

\vspace{-2mm}
\end{frame}

\subsection{机制分析 I}\label{ux673aux5236ux5206ux6790-i}

\begin{frame}{机制分析 I}
\begin{adjustbox}{max width=0.95\textwidth,center}
\begin{table}[htbp]
\footnotesize
\centering
\caption{税收征管力度与财政支出的创业驱动效应}\label{tab:mechanism}
\begin{threeparttable}
\begin{tabular}{l*{4}{c}}
\toprule
 & \multicolumn{4}{c}{\textbf{因变量}} \\
\cmidrule(lr){2-5}
变量 & (1) 税收征管力度 & (2) 生产性支出 & (3) 民生性支出 & (4) 县域创业活力 \\
\midrule
省直管县 & 0.031 & 0.161\tnote{**} & 0.028\tnote{*} & 0.057\tnote{**} \\
 & (0.022) & (0.084) & (0.015) & (0.037) \\
\addlinespace[2pt]
税收征管力度 & — & — & — & 0.008 \\
 &  &  &  & (0.025) \\
生产性支出 & — & — & — & 0.015\tnote{***} \\
 &  &  &  & (0.005) \\
民生性支出 & — & — & — & 0.050\tnote{*} \\
 &  &  &  & (0.029) \\
\midrule
时间趋势 & 控制 & 控制 & 控制 & 控制 \\
固定效应 & 个体、时间 & 个体、时间 & 个体、时间 & 个体、时间 \\
观测数 & 12,202 & 12,202 & 12,202 & 12,202 \\
R平方 & 0.736 & 0.802 & 0.941 & 0.903 \\
\bottomrule
\end{tabular}

\begin{tablenotes}
\scriptsize
\item 注:括号内为县级聚类稳健标准误。***、**、*分别表示1%、5%、10%显著性水平。
\item 时间趋势包含:人均GDP、人口规模等社会经济变量的二次项趋势。
\item 财政支出口径调整导致样本期间为2000-2006年,数据来源为《全国地市县财政统计资料汇编》。
\end{tablenotes}
\end{threeparttable}
\end{table}

\end{adjustbox}

\scriptsize

民生性支出的边际效应超过生产性支出94\%,验证财政支出结构调整的重要性
\end{frame}

\subsection{机制分析 II}\label{ux673aux5236ux5206ux6790-ii}

\begin{frame}{机制分析 II}
\begin{adjustbox}{max width=0.95\textwidth,center}
\begin{table}[!htbp]
\footnotesize
\centering
\caption{省直管县的财政治理方式影响}\label{tab:mechanism}
\begin{threeparttable}
\begin{tabular}{lccccc}
\toprule
变量 & (1) & (2) & (3) & (4) & (5) \\
 & 税收分成比例 & 人均转移支付 & 税收征管力度 & 生产性支出 & 民生性支出 \\
\midrule
省直管县 & 0.014\tnote{**} & 0.042\tnote{***} & 0.027 & 0.126 & 0.015 \\
 & (0.006) & (0.015) & (0.022) & (0.079) & (0.014) \\
\addlinespace[0.5ex]
税收分成比例 & & & 0.476\tnote{***} & -0.183 & 0.189\tnote{***} \\
 & & & (0.048) & (0.179) & (0.033) \\
\addlinespace[0.5ex]
人均转移支付 & & & -0.056\tnote{***} & 0.884\tnote{***} & 0.248\tnote{***} \\
 & & & (0.014) & (0.064) & (0.016) \\
\midrule
观测数 & 12,203 & 12,187 & 12,186 & 12,187 & 12,187 \\
R平方 & 0.854 & 0.966 & 0.744 & 0.809 & 0.945 \\
控制变量 & \multicolumn{5}{c}{时间趋势项、个体固定效应} \\
固定效应 & \multicolumn{5}{c}{个体、时间双重固定效应} \\
\bottomrule
\end{tabular}

\begin{tablenotes}
\scriptsize
\item 注:括号内为县级聚类标准误,***、**、*分别表示1%、5%、10%显著性水平。
\item 样本区间为2000-2006年,覆盖全国县级财政单位。
\item 第(1)(2)列为第一阶段回归,展示改革对税收分成比例和转移支付的直接影响。
\item 第(3)-(5)列第二至第四阶段回归,检验税收分成和转移支付的中介效应。
\end{tablenotes}
\end{threeparttable}
\end{table}

\end{adjustbox}

\scriptsize

民生性支出的边际效应超过生产性支出94\%,验证财政支出结构调整的重要性
\end{frame}

\subsection{异质性分析:市县经济实力}\label{ux5f02ux8d28ux6027ux5206ux6790ux5e02ux53bfux7ecfux6d4eux5b9eux529b}

\begin{frame}{异质性分析:市县经济实力}
\begin{adjustbox}{max width=0.95\textwidth,center}
\begin{table}[htbp]
\footnotesize
\centering
\caption{市县经济实力异质性分析结果}\label{tab:hetero_eco}
\begin{threeparttable}
\begin{tabular}{lcccc}
\toprule
 & \multicolumn{4}{c}{\textbf{县域创业活力}} \\
\cmidrule(lr){2-5}
变量 & (1) & (2) & (3) & (4) \\
 & 市GDP & 市排名 & 县GDP & 县排名 \\
\midrule
省直管县 & 0.526\tnote{**} & 0.052 & 0.856\tnote{***} & 0.007 \\
 & (0.220) & (0.029) & (0.186) & (0.027) \\
省直管县$\times$经济指标 & -0.051\tnote{**} & 0.059 & -0.093\tnote{***} & 0.153\tnote{***} \\
 & (0.025) & (0.044) & (0.022) & (0.045) \\
\midrule
观测数 & 24,297 & 24,297 & 24,297 & 24,297 \\
R平方 & 0.896 & 0.896 & 0.896 & 0.896 \\
\bottomrule
\end{tabular}

\begin{tablenotes}
\scriptsize
\item[注]:括号内为县级聚类标准误,***、**、*分别表示1%、5%、10%显著性水平。
\item (1)(2)列考察市级经济实力调节效应,(3)(4)列考察县级经济实力调节效应。
\item 经济指标归一化处理: 市/县GDP取对数,排名为省级百分位数。
\end{tablenotes}
\end{threeparttable}
\end{table}

\end{adjustbox}
\end{frame}

\subsection{异质性分析:市场化程度}\label{ux5f02ux8d28ux6027ux5206ux6790ux5e02ux573aux5316ux7a0bux5ea6}

\begin{frame}{异质性分析:市场化程度}
\begin{adjustbox}{max width=0.95\textwidth,center}
\begin{table}[!htbp]
\footnotesize
\centering
\caption{市场化程度异质性结果}\label{tab:hetero_market}
\begin{threeparttable}
\begin{tabular}{lcccc}
\toprule
 & (1) & (2) & (3) & (4) \\
 & 创业活力 & 税收征管 & 生产支出 & 民生支出 \\
\midrule
省直管县 & 0.314\tnote{***} & -0.114 & 0.040\tnote{*} & -0.058 \\
 & (0.070) & (0.089) & (0.016) & (0.103) \\
市场化$\times$省直管县 & -0.053\tnote{**} & 0.031 & -0.006 & 0.029 \\
 & (0.015) & (0.019) & (0.003) & (0.022) \\
\addlinespace[0.5ex]
观测数 & 24,297 & 12,202 & 12,203 & 12,203 \\
R平方 & 0.896 & 0.903 & 0.903 & 0.903 \\
\bottomrule
\end{tabular}

\begin{tablenotes}
\scriptsize
\item 注:***、**、*分别表示1%、5%、10%显著性水平;括号内为县级聚类标准误
\item 市场化指数测度采用樊纲市场化指数(2000-2003年均值)
\end{tablenotes}
\end{threeparttable}
\end{table}

\end{adjustbox}

\begin{itemize}
\tightlist
\item
  市场化程度显著弱化省直管县改革的创业激励效应
\end{itemize}
\end{frame}

\subsection{异质性分析:企业类型}\label{ux5f02ux8d28ux6027ux5206ux6790ux4f01ux4e1aux7c7bux578b}

\begin{frame}{异质性分析:企业类型}
\begin{adjustbox}{max width=0.95\textwidth,center}
\begin{table}[!htbp]
\footnotesize
\centering
\caption{企业规模及类型异质性分析结果}\label{tab:hetero_firm}
\begin{threeparttable}
\begin{tabular}{l*{6}{c}}
\toprule
 & \multicolumn{3}{c}{\textbf{企业规模}} & \multicolumn{3}{c}{\textbf{企业类型}} \\
\cmidrule(lr){2-4} \cmidrule(l){5-7}
变量 & (1) & (2) & (3) & (4) & (5) & (6) \\
 & <500万 & ≥500万 & 中小企业占比 & 非个体户 & 个体户 & 个体户占比 \\
\midrule
省直管县 & 0.005 & 0.089\tnote{***} & -0.014\tnote{**} & 0.066\tnote{***} & -0.053 & -0.013\tnote{**} \\
 & (0.021) & (0.019) & (0.004) & (0.015) & (0.041) & (0.006) \\
\addlinespace[1ex]
观测数 & \multicolumn{3}{c}{24,297} & \multicolumn{3}{c}{24,297} \\
R平方 & 0.812 & 0.855 & 0.609 & 0.898 & 0.790 & 0.691 \\
\bottomrule
\end{tabular}

\begin{tablenotes}
\scriptsize
\item 注:***、**、*分别表示1%、5%、10%显著性水平,括号内为县级聚类标准误。
\item 控制变量包含时间趋势项、事前社会经济特征及省份固定效应(受限版面未展示)。
\end{tablenotes}
\end{threeparttable}
\end{table}

\end{adjustbox}

\begin{itemize}
\tightlist
\item
  省直管县改革更显著促进大型企业和非个体工商户的创业,而中小企业及个体户的激励效应较弱
\end{itemize}
\end{frame}

\end{document}
